\documentclass[10pt]{aleph-libro}

% -- Paquetes adicionales
\usepackage{enumitem}
\usepackage{amssymb}

\usepackage{lipsum}

\graphicspath{{./Ejemplo/}}

% -- Datos del libro
\autor[A. Merino]{Andrés Merino}
\titulo{Matemática para diseño}
\subtitulo[:]{Herramientas básicas}
%\fasciculo{Número reales}
\numero{1 (1)}
\serie{Cuadernos de Matemática\\[1mm] Escuela de Ciencias}{Cuaderno de matemática de la Escuela de Ciencias}
\editor{Andrés Merino}
%\revision{}
%\asistente{}
\fechapub{2023}
\edicion{Segunda}{2023}
\impresion{Primera}{2018}
\registroautoral{}
\ISBN{978-0-00000-000}
\publicado{en linea por Andrés Merino,\par Quito, Ecuador.}
\derechos{Andrés Merino}
\nota{Queda pro}

% -- Logos
\logouno{Logos/logo03}{3cm}{2.5cm}
\logodos{Logos/logo01}{3cm}{2.5cm}
\logofondo{Logos/logoFondo}

% -- Colores
\definecolor{colorp}{cmyk}{0.81,0.62,0.00,0.22}
\definecolor{colordef}{cmyk}{0.81,0.62,0.00,0.22}

% -- Otras adaptaciones
\ecuadroblanco{.32\paperwidth}

% -- Fuentes
\fuente{montserrat}

%%%%%%%%%%%%%%%%%%%%%%%%%%%%%%%%%%%%%%%%
%%  Ambientes en estilo clásico
%%%%%%%%%%%%%%%%%%%%%%%%%%%%%%%%%%%%%%%%
% - Ambientes con recuadro sin titulo aparte
\theoremstyle{estiloteorema}
    \newtheorem{pcuno}[prop]{Personalizado Uno}
        \tcolorboxenvironment{pcuno}{color=brown,recuadrost}

% - Ambientes con título aparte con otra numeración
\newcounter{pnum}[chapter]
\renewcommand{\thepnum}{\thechapter.\arabic{pnum}}
\newtcolorbox{pcdos}[1][]
    {tipo=Personalizado Dos,contador=pnum,color=magenta,recuadroctizq={#1}}

% - Ambientes con título aparte con numeración y a la derecha
\newtcolorbox{pctres}[1][]
    {tipo=Personalizado Tres,contador=pnum,color=green,recuadroctder={#1}}

%%  Ambientes con formato de advertencia.
\newtcolorbox{pccuatro}
    {icono=\faCloudDownload,color=pink,postit}
    
%%%%%%%%%%%%%%%%%%%%%%%%%%%%%%%%%%%%%%%%
%%  Ambientes en estilo nuevo
%%%%%%%%%%%%%%%%%%%%%%%%%%%%%%%%%%%%%%%%
% - Ambientes sin recuadro
\theoremstyle{estiloteoreman}
    \newtheorem*{pncero}{\color{red} \tikz \fill (1ex,1ex) circle (3.5pt); Personalizado cero}
        
% - Ambientes con recuadro sin titulo aparte
\theoremstyle{estiloteoreman}   
    \newtheorem{pnuno}[prop]{Personalizado Uno}
        \tcolorboxenvironment{pnuno}{%
            color=brown,recuadrost,colback=red!10,drop fuzzy shadow
        }

% - Ambientes con título aparte con otra numeración
\newcounter{pnumnn}[chapter]
\renewcommand{\thepnumnn}{\thechapter.\arabic{pnumnn}}
\newtcolorbox{pndos}[1][]
    {tipo=Personalizado Dos,contador=pnumnn,color=magenta,recuadroctint={#1}}

\newtcolorbox{pntres}[1][]
    {tipo=Personalizado Tres,contador=pnumnn,color=olive,recuadroctint={#1},
    colback=lime,colbacktitle=lime,colframe=lime}

\begin{document}
\portada
\portadilla
\tabladecontenidos
\mainmatter

\chapter{Números Reales}    

\section{Propiedades de los números reales}

\section{Proporciones y porcentajes}

\section{Ecuaciones lineales}

\section{Ecuaciones no lineales}

\section{Inecuaciones lineales}

Hola
    \cleartooddpage[\thispagestyle{empty}]

\chapter{Ejemplos}

\begin{defi}[Coordenadas polares]
    En $\mathbb{R}^2$, el cambio a coordenadas polares ...
\end{defi}

\begin{advertencia}
    Notemos que si $(x,y)=P(r,\theta)$, entonces ...
\end{advertencia}

\begin{teo}
    Teorema...
\end{teo}

\begin{teo}[Título del teorema]
    Teorema...
\end{teo}

\begin{cor}
    Corolario...
\end{cor}

\begin{cor}[Título]
    Corolario...
\end{cor}

\begin{lem}
    Lema...
\end{lem}

\begin{obs}
    Lorem ipsum dolor sit amet, consectetur adipiscing elit. Sed venenatis massa vitae dui auctor ornare. Quisque fermentum ex ligula. Fusce eget placerat turpis, a gravida quam. Nullam sit amet neque dignissim, dignissim est at, mollis sem. Pellentesque vulputate malesuada libero, ac porttitor massa mattis sit amet. Etiam aliquet consequat iaculis.
\end{obs}

\begin{ejem}
    Lorem ipsum dolor sit amet, consectetur adipiscing elit. Sed venenatis massa vitae dui auctor ornare. Quisque fermentum ex ligula. Fusce eget placerat turpis, a gravida quam. Nullam sit amet neque dignissim, dignissim est at, mollis sem. Pellentesque vulputate malesuada libero, ac porttitor massa mattis sit amet. Etiam aliquet consequat iaculis.
\end{ejem}

\begin{ejer}
    Calcular \[\iint_D x^2+y^2\,dxdy\] donde $D=\{(x,y\in\mathbb{R}^2:x^2+y^2\leq 4,\ y>0\}$.
\end{ejer}

\begin{proof}[Solución]
    Para resolver esta integral...
\end{proof}

%%%%%%%%%%%%%%%%%%%%%%%%%%%%%%%%%%%%%%%%
\newpage

\begin{pncero}
    Lorem ipsum dolor sit amet, consectetur adipiscing elit. Maecenas commodo lacus lectus, vitae imperdiet nulla tincidunt at. Nam viverra augue orci, nec efficitur nunc luctus non. Proin tincidunt, risus id accumsan molestie, ipsum orci interdum augue, lacinia finibus nunc ante ut lacus.
\end{pncero}

\begin{pnuno}
    Lorem ipsum dolor sit amet, consectetur adipiscing elit. Maecenas commodo lacus lectus, vitae imperdiet nulla tincidunt at. Nam viverra augue orci, nec efficitur nunc luctus non. Proin tincidunt, risus id accumsan molestie, ipsum orci interdum augue, lacinia finibus nunc ante ut lacus.
\end{pnuno}

\begin{pndos}[Título]
    Lorem ipsum dolor sit amet, consectetur adipiscing elit. Maecenas commodo lacus lectus, vitae imperdiet nulla tincidunt at. Nam viverra augue orci, nec efficitur nunc luctus non. Proin tincidunt, risus id accumsan molestie, ipsum orci interdum augue, lacinia finibus nunc ante ut lacus.
\end{pndos}

\begin{pntres}
    Lorem ipsum dolor sit amet, consectetur adipiscing elit. Maecenas commodo lacus lectus, vitae imperdiet nulla tincidunt at. Nam viverra augue orci, nec efficitur nunc luctus non. Proin tincidunt, risus id accumsan molestie, ipsum orci interdum augue, lacinia finibus nunc ante ut lacus.
\end{pntres}
    \cleartooddpage[\thispagestyle{empty}]

\contraportada{Escuela de Ciencias Físicas y Matemática}
   {https://en.wikipedia.org/wiki/QR_code}
   {\lipsum[7]}


\end{document}
